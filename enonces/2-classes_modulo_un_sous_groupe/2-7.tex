Soit $\mathcal I (2)$ le groupe des isométries du plan affine euclidien $P$, on pose 

\[\mathcal{D}(2) = \mathcal{T}(P) \cup \left( \bigcup_{O \in P} \mathcal{R} (P,O) \right).\]

En utilisant les résultats de l'exercice 26 de connard, démontrer que $\mathcal D (2)$ est l'ensemble des isométries pouvant s'écrire comme produit d'un nombre pair de symétries; en déduire que $\mathcal D (2)$ est un sous-groupe de $\mathcal I (2)$ tel que $\indice {\mathcal I (2)} {\mathcal D (2)} = 2$. $\mathcal D (2)$ est appelé le groupe des déplacements de $P$.