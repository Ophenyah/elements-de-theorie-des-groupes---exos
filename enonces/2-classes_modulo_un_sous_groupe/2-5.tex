Soit $H$ et $K$ deux sous-groupes (non nécessairement distincts) d'un groupe $G$. On considère la relation binaire $\mathcal{R}_{H,K}$ définie dans $G$ par :
\[
x \mathcal{R}_{H,K}\ y \Leftrightarrow \exists (h, k) \in H \times K,\ y=hxk.
\]

\begin{abc}
\item 
Vérifier que $\mathcal{R}_{H,K}$ est une relation d'équivalence dans $G$ et que la classe modulo $\mathcal{R}_{H,K}$ d'un élément $x \in G$ est $HxK = \left\{ hxk; (h,k) \in H \times K \right\}$.

$HxK$ sera appelée "classe double de $x$ modulo $H$ et $K$".

\item $x$ étant donné dans $G$, vérifier que l'application 

\deffun{\lambda}{HxK}{x^{-1}HxK}{hxk}{x^{-1}hxk}

est une bijection.

\item On suppose que le groupe $G$ est fini; soit $r \in \N^*$ le nombre de classes doubles distinctes, de $G$ modulo $H$ et $K$; on les notera $Hx_iK, 1 \leq i \leq r$.

Pour tout $(1 \leq i \leq r)$, justifier les propriétés suivantes :

\begin{enumerate}[label=\greek*)]
\item  $|Hx_iK| = |x_i^{-1}Hx_iK|$ (égalité des cardinaux);
\item $x_i^{-1}Hx_i$ est un sous-groupe de $G$ et $\sigma(x_i^{-1}Hx_i) = \sigma(H)$;
\item $|Hx_iK| = \dfrac{\sigma(H) \sigma(K)}{\sigma(x_i^{-1}Hx_i \cap K)}$ (voir l'exercice 3 ci-dessus).
\end{enumerate}
En déduire la relation :

\[\sigma(G) = \sigma(H)\sigma(K) \displaystyle \sum_{i=1}^r d_i^{-1} \]

où, pour tout $i\ (1 \leq i \leq r)$, $d_i = \sigma(x_i^{-1}Hx_i \cap K)$.

\item D'après les résultats précédents, a priori, dans un groupe $G$, deux classes doubles distinctes modulo H et K, ne sont pas, en général, équipotentes; montrer que la décomposition du groupe symétrique $S_3$ en classes doubles modulo $H \langle \tau_1 \rangle$ et $K = \langle \tau_2 \rangle$ confirme cette remarque et permet aussi de vérifier que $H \neq K$ implique, en général dans un groupe non abélien, $\mathcal{R}_{H,K} \neq \mathcal{R}_{K,H}$.

\end{abc}