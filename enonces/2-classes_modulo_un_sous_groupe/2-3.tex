Soient $H$ et $K$ deux sous-groupes finis d'un groupe $G$. On pose $[K : H \cap K ] = n$. Soit $\{ x_i \}_{1 \leq i \leq n}$ une famille de représentants des classes à droite de distinctes de  $K$ modulo $H \cap K$.

\begin{abc}
\item  Démontrer que les  $\{ Hx_i \}_{1 \leq i \leq n}$ forment une partition de l'ensemble $HK$ (qui n'est pas nécessairement un sous-groupe).
\item $|HK|$ et $|KH|$ désignant les cardinaux de HK et KH, prouver que 

\[ |HK| = |KH| = \dfrac{\sigma(H) \sigma(K)}{\sigma(H \cap K)}.\]
\item Lorsque $HK$ est un sous-groupe de $G$, vérifier que $\sigma(HK)$ donné par la formule précédente, peut aussi être obtenu à partie du résultat b) de l'exercice 2 ci-dessus.
\end{abc}