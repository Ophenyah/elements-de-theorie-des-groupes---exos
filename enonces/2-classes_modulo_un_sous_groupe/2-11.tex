Soit $G$ un groupe de type fini, et $H$ un sous-groupe d'indice fini dans $G$; le but de l'exercice est de prouver que $H$ est de type fini.

On pose $G = \langle x_1, x_2, \ldots , x_n \rangle, n \leq 1$ dans $\N$. et $\indice G H = m \leq 1$ dans $\N$. 

\begin{enumerate}
\item  Vérifier les propriétés suivantes :

\begin{abc}
\item Si, pour tout $i (1 \leq i \leq n)$, on pose $x_{n+i} = x_i^{-1}$, alors $G$ peut être considéré comment engendré par $\{x_1,\ldots,x_n,x_{n+1},\ldots,x_{2n}\}$.
\item On pose $G = \displaystyle \bigcup_{1 \leq j \leq m} Hg_j$, $\{g_j\}_{1 \leq j \leq n}$ étant une famille de représentants des classes à droite de $G$ modulo $H$, avec $g_1=e$; alors pour tout couple $(i,j), 1\leq i \leq 2n$, $1 \leq j \leq m$, il existe un unique $h_{ij} \in H$ et un unique entier $k (1 \leq k \leq m)$ tels que :

\[g_ix_i = h_{ij}g_k.\]
\end{abc}

\item Compte tenu des notations ci-dessus, démontrer que $H$ est engendré par $\{h_{ij}; 1\leq i \leq 2n$, $1 \leq j \leq m\}.$
\end{enumerate}