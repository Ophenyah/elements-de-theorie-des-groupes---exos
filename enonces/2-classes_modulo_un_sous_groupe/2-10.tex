Soient $G$ un groupe et $H$ un sous-groupe d'indice fini dans $G$. pour tout $g \in G$, $HgH$ sera appelé : classe double de $G$ modulo $H$.

\begin{abc}
\item Montrer que le nombre de classes doubles de $G$ modulo $H$ est fini.
\item  Etant donné $g \in G$, vérifier que l'on peut écrire :

\[HgH = \displaystyle
\bigcup_{1 \leq i \leq p} Hgx_i, \text{ où } p = \indice H {H \cap g^{-1}Hg},
\]

$x_i \in H$ pour tout $i (1 \leq i \leq p)$ et $Hgx_i \cap Hgx_j = \varnothing$, si $i \neq j$.

En utilisant l'exercice 9 si dessus, prouver que l'on A
\[
HgH = \displaystyle \bigcup_{1 \leq i \leq p} y_j g H,
\]

où $y_i \in H$ pour tout $j (1 \leq j \leq p)$ et $y_i gH \cap y_j g H = \varnothing$, si $i \neq j$.

on pose $z_i = y_i g x_i$, vérifier que 

\[
HgH = \displaystyle \bigcup_{1 \leq i \leq p} Hz_i = \bigcup_{1 \leq i \leq p} z_i H.
\]

\item En conclusion de ce qui précède, prouver que, si $\indice G H = r$, il existe dans $G$ des éléments de classes $a_1,a_2,\ldots, a_r$ formant une famille de représentants, à la fois de l'ensemble des classes à droite et de l'ensemble des classes à gauche modulo $H$.
 \end{abc}