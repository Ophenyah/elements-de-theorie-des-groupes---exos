Soit $\R $ considéré comme un espace affine euclidien de dimension 1, muni de la distance habituelle $d$ telle que $d(x,y) = |x-y|$.

A tout $a \in \R$ on associe les translations :

\deffun {\tau_a} \R \R x {x+a}

\deffun {\text{et } \sigma_a} \R \R x {a-x} 

\begin{abc}
\item Vérifier que, quels que soient $a,b$ dans $\R$, $\sigma_a^2 = id_\R$ et $\sigma_b \circ \tau_a = \tau_{-a} \circ \sigma_b$.
\item Montrer que sur la droite réeele, $\sigma_a$ est une symétrie par rapport à un point.
\item Démontrer que toute isométrie de $\R$ est soit une translation, soit une symétrie par rapport à un point. En déduire que l'ensemble $\mathcal{I}(1)$ des isométries de $\R$ est un sous-groupe non abélien du groupe symétrique $S_\R$.
\item $T = \{\tau_a; a \in \R\} $ étant un sous-groupe abélien de $S_\R$, montrer que $\indice {\mathcal{I}(1)} T =2$.
\end{abc}