Soit $G$ un ensemble non vide muni d'une loi de composition interne \textit{associative} notée $\cdot$ : on suppose que dans $(G,\cdot)$ les deux conditions suivantes sont vérifiées : 
\begin{enumerate}[label=\arabic*°]
    \item il existe un élément \textit{neutre à droite e} (voir exercice 1) ;
    \item tout élément $x \in G$ admet un \textit{symétrique à droite}, $x'$ (voir exercice 1).
\end{enumerate}

Démontrer que $(G,\cdot)$ est un groupe; vérifier, par un contre exemple, que, sans l'associtivité de la loi $\cdot$, ce résultat n'est plus vrai.