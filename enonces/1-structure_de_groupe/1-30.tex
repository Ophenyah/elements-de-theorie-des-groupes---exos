 $\R$ désignant le groupe additif des réels, on pose :

 \[
 J = \left\{ x \in \R; 0 \leq x \leq 1\right\}.
 \]

 L'addition de $\R$ induit dans l'ensemble $\R^J$ une structure de groupe additif abélien.
 
\begin{abc}

    \item  Vérifier les propriétés suivantes :
    \begin{itemize}
        \item l'ensemble des fonctions $f \in \R^J$, continues sur $J$, est un sous-groupe de $(\R^J,+)$, que l'on notera $\mathcal{C}(J)$;
        \item si, pour tout $a \in \R$, on note $c_a$ la fonction constante de $J$ dans $\R$  telle que $c_a(x) = a$ pour tout $x \in J$, alors 
        $\Gamma = \left\{ c_a; a \in \R \right\}$ est un sous-groupe de $(\mathcal{C}(J), +)$.
    \end{itemize}

    \item On considère les applications $F_i$ de $\mathcal{C}(J)$ dans $\R$ telles que :
    \[
    F_1 : f \mapsto f(1),\quad F_2 : f \mapsto |f(0)|,\quad F_3 : f \mapsto \int_0^1 f(x)dx
    \]

    \[
    F_4 : f \mapsto \dfrac{\pi}{3} \int_0^1 f(x) \cos \dfrac{\pi x}{6} dx,\quad F_5 : f \mapsto \int_0^1 cos \dfrac{\pi f(x)}{6} dx.
    \]
Déterminer les $F_i$ qui sont des homomorphismes de groupes de $(\mathcal{C}(J), +)$ dans $(\R, +)$. Pour chacun des morphismes de groupes $F_i$, prouver que, quel
que soit $a \in \R,\ F_i(c_a) = a$ et montrer qu'il existe un unique $m_i \in \R$ tel que $F_i(id_J - C_{m_i}) = 0$. En déduire que les $\ker F_i$ sont deux à deux distincts.

    
\item Démontrer que pour tout $F \in Hom(\mathcal{C}(J), \R)$, tel que $F(c_a) = a$, quel que soit $a \in \R$, on a 
\[
\mathcal{C}(J) = \ker F \oplus \Gamma.
\]

En conclure qu'il existe de nombreux sous-groupes de $\mathcal{C}(J)$ tels que $\mathcal{C}(J) = H \oplus \Gamma$.
\end{abc} 
