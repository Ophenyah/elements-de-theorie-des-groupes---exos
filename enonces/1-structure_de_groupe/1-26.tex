Soit P le plan affine euclidien. Si $f$ est une isométrie du plan P, on dit qu'un point A est fixe pour $f$ si $f(A)=A$. 

On désigne par $\mathcal{I}(2)$ l'ensemble des isométries du plan P. 

Si $\Delta$ est une droite de P, on note $s_\Delta$ la symétrie du plan par rapport à $\Delta$; \inlinedeffun{s_\Delta}{P}{P}{A}{A'}, $A'$ est tel que $\Delta$ est la médiatrice de $AA'$. 


\begin{abc}
    \item Vérifier les propriétés suivantes :
    \begin{itemize}
        \item L'identité de P, notée $id_P$, appartient à $\mathcal{I}(2)$.
        \item quelle que soit la droite $\Delta$, $s_\Delta$ appartient à $\mathcal{I}(2)$ et $s_\Delta \circ s_\Delta = id_P$.
        \item Si $f_1$ et $f_2$ sont dans $\mathcal{I}(2)$, alors $f_2 \circ f_1 \in \mathcal{I}(2)$; $f_2 \circ f_1$ sera appelé le produit de $f_1$ et $f_2$ dans $\mathcal{I}(2)$.     
    \end{itemize}
\item Soit $f \in \mathcal{I}(2)$; montrer que :
\begin{itemize}
    \item si $f$ à deux points fixes distincts A et B, alors tout point de la droite AB est fixe pour $f$;
    \item Si $f$ à trois points fixes, A, B, C non alignés, alors $f = id_P$.
\end{itemize}
\item Démontrer que toute isométrie $f \in \mathcal{I}(2)$ est le produit de 0, 1, 2, ou 3 symétries.
\item Prouver que $\mathcal{I}(2)$ est un sous-groupe du groupe symétrique $S_p$ et que $\mathcal{I}(2)$ est non-abélien.
\item A tout vecteur $v$  de l'espace vectoriel $\R^2$ on associe la translation de vecteur $v$ du plan affine P, notée $t_v$.
Montrer à l'aide de (c) que $t_v \in \mathcal{I}_2$ et que $\mathcal{T}(P) = \left\{ t_v;\ v \in \R^2 \right\}$ est un sous-groupe abélien de $\mathcal{I}(2)$, isomorphisme à $(\R^2,+)$. 

\item Soit O un point du plan P, pour $\alpha \in \R$; on note $r_{O, \alpha}$ la rotation du plan P de centre O et d'angle $\alpha$.

Montrer à l'aide de (c) que $r_{O, \alpha} \in \mathcal{I}(2)$. $\mathcal{R}(P,O)$ désignant l'ensemble de toutes les rotations $R_{O,\alpha}$ pour $\alpha \in \R$, 
vérifier que $\mathcal{R}(P,O) = \left\{ r_{O, \alpha};\ 0 \leq \alpha < 2\pi \right\}$ et que $\mathcal{R}(P,O)$ est un sous-groupe abélien de $\mathcal{I}(2)$.
\end{abc}