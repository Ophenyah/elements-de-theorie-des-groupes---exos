Soit $n>1$ dans $\N$ et $\left( \dfrac{\Z}{(n)},+ \right)$ le groupe des classes de congruence modulo $n$. On considère la correspondance $\mu$ définie par :

\deffun{\mu}{ \dfrac{\Z}{(n)} \times  \dfrac{\Z}{(n)}}{ \dfrac{\Z}{(n)}}{(\overline{x},\overline{y})}{ \overline{xy} }

\begin{abc}
\item Prouver que la correspondance $\mu$ est une application [c'est-à-dire que : $(\overline{x'}=\overline{x} \text{ et } \overline{y'} = \overline{y} \Rightarrow \overline{x'y'} = \overline{xy} )$].

En déduire que l'on peut définir dans $\dfrac{\Z}{(n)}$ une multiplication telle que $\overline{x} \cdot \overline{y} = \overline{x \cdot y}$.

Montrer alors que $\dfrac{\Z}{(n)}$ est un anneau unitaire. et commutatif.

\item  Soit, dans $\N$, un nombre premier $p$. On désigne par $G_p$ l'ensemble des éléments non nuls de $\dfrac{\Z}{(p)}$.

Prouver, en utilisant le résultat de l'exercice 4, que $G_p$ est un groupe par rapport à la multiplication définie dans $\dfrac{\Z}{(p)}$.

En conclure que $\dfrac{\Z}{(p)}$ est un corps.

\item Vérifier que si $n$ n'est pas premier $\dfrac{\Z}{(p)}$ n'est pas un corps.
\end{abc}