Soit E un ensemble non vide et $G$ un groupe d'élément unité $e$. On désigne par $G^E$ l'ensemble des applications $f$ de $E$ dans $G$.
 On considère la loi de composition définie dans $G^E$ par :

 \begin{align*}
    G^E \times G^E &\rightarrow G^E \\
    (f,g) &\mapsto fg,
 \end{align*}

 Où $fg$ est telle que pour tout $x \in E$, $(fg)(x) = f(x)g(x)$.

 Prouver que $(G^E)$ est ainsi muni d'une structure de groupe.

 Vérifier que $G^E$ est un groupe abélien si et seulement si $G$ est abélien.