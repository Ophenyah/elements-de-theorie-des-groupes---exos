
\begin{enumerate}
    \item $\gpn$ est le groupe des générateurs $\cycle{p^n}$, il est donc abélien d'ordre $\varphi(p^n) = p^{n-1}(p-1)$.
    \item $G_9$ est constitué des éléments relativement premiers avec 9, donc $ G_9 = \{1,2,4,5,7,8\}$. On vérifie que $G_9 = \langle 2 \rangle = \langle 5 \rangle$, avec 2 et 5 les seuls générateurs.
       \item \begin{enumerate}[label=\alph*)]
            \item  Soient x et y tel que $\overline{x} = \overline{y}$,  il existe $k\in \mathbb{Z}$ tel que $x=y+kp^n$, ainsi :
    \begin{align*}
        \overline{x}=\overline{y} \Rightarrow \varphi(\overline{x}) &= \Dot{x}, \\
        &= \Dot{(y + kp^n)}, \\
        &= \Dot{y}, \\
        &= \varphi(\overline{y}).
    \end{align*}
    $\varphi$ est une application qui préserve l'équivalence. Montrons que c'est un morphisme, pour tout $\overline{x}$ et $\overline{y}$ dans $\gpn$ : 
    \begin{align*}
        \varphi(\overline{x}\overline{y}) &= \varphi(\overline{xy}), \\
        &= \Dot{(xy)}, \\
        &= \Dot{x}\Dot{y}, \\
        &= \varphi(\overline{x})\varphi(\overline{y}).
    \end{align*}
Pour n'importe quel élément $x$ de $G_p$, on a $pgcd(x,p)=1$ et directement $pgcd(x,p^n)=1$.
        \item 
        \item 
    \end{enumerate} 
    \item     \begin{enumerate}[label=\alph*)]
        \item 
        \item 
        \item
    \end{enumerate}
    \item
    \item 
    \begin{enumerate}[label=\alph*)]
        \item 
        \item
    \end{enumerate}
\end{enumerate}
