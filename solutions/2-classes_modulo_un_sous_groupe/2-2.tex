\begin{abc}

\item  

\begin{align*}
    Kx_i = Kx_j &\Rightarrow x_ix_j^{-1} \in K, \\
    &\Rightarrow x_ix_j^{-1} \in H \inter K \text{ car x famille de H modulo } H \inter K, \\
    &\Rightarrow (H \inter K)x_i = (H \inter K)x_j, \\
    &\Rightarrow i = j.
\end{align*}

La réciproque étant triviale, on a bien $Kx_i = Kx_j \Leftrightarrow i = j$.

Un représentant d'une classe à droite de $H$ modulo $H \inter K$ est aussi un réprésentant d'une classe à droite de $G$ modulo $K$, on en déduit que :

\[ \indice H {H \inter K} 
\leq
\indice G K
<
\infty.
\]
 
On a démontré que $G \supseteq \displaystyle \bigcup_{i \in I} Kx_i $.

Soit $g \in G$, par hypothèse, $\exists (h,k) \in H \times K$ tel que $g = hk$. Il existe $i \in I$ tel que $h \in (H \inter K)x_i \subseteq Kx_i$, et donc $G =  \displaystyle \bigcup_{i \in I} Kx_i.$

\item 

D'après la formule des indices, on a :

\[\indice G {H \inter K} = \indice G H \indice H {H \inter K} \leq \indice G H \indice G K.\]

L'égalité quand $G=HK$ découlant directement du point précédent.

\end{abc}