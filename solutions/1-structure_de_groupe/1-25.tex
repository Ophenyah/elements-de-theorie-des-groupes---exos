\begin{abc}
\item On a démontré que $\Gamma_1 \simeq \Gamma_2 \simeq \Gamma_3$. $\Gamma_1$ étant un sous-groupe de $GL(2,\R)$, $\Gamma_2$ et $\Gamma_3$ admettent un morphisme bijectif pour un sous-groupe $GL(2,\R)$, qui est aussi un morphisme injectif dans $GL(2,\R)$. En appliquant le même raisonement pour $K_2$ et $S_3$ , on trouve que ces 4 groupes admettent une représentation matricielle de degré 2 sur $\R$.
\item Posons l'application $\varphi$ tel que :

\deffun{\varphi}{\C^*}{GL(2,\R)}{a+ib}{\begin{pmatrix}
    a & -b \\
    b & a 
\end{pmatrix}}

Vérifions tout d'abord que toute image est bien dans $GL(2, \R)$ :

\begin{align*}
   \forall a+ib \in \C, det(\varphi(a+ib)) = a^2 + b^2 \neq 0. 
\end{align*}


Vérifions que c'est un morphisme :

\begin{align*}
    \forall a+ib, c+id \in \C^*, \varphi((a+ib) + (c +id)) &= \varphi( (a+c) + i(b+d) ), \\
    &= \begin{pmatrix}
        a+c & -(b+d) \\
        b+d & a+c
    \end{pmatrix}, \\
    &=\begin{pmatrix}
        a & -b \\
        b & a 
    \end{pmatrix}
    \begin{pmatrix}
        c & -d \\
        d & c
    \end{pmatrix}, \\
    &\varphi(a+ib)\varphi(c+id).
\end{align*}

Enfin, vérifion son injectivité : 

\begin{align*}
    \ker(\varphi) &= \left\{ a+ib \in C^*, \varphi(a+ib) = I_2  \right\}, \\
    &= \left\{ a+ib \in C^* , a=1 \land b=0  \right\}, \\
    &= \left\{ 1 \right\}.
\end{align*}

Ainsi,  le groupe multiplicatif $C^*$ admet une représentation matricielle de degré 2 sur $\R$.

\end{abc}