\begin{abc}

\item  Démontrons par récurrence sur $n$.

\begin{itemize}
    \item Initialisation : 

    Soit $H_1, H_2$ deux sous-groupes de $G$ tel que $H_1H_2$ est un sous-groupe de $G$, la propriété est vérifiée.

    \item Hérédité : 
    
    Supposons qu'il existe $n$ tel que la propriété est vraie, c'est à dire que pour $\left\{ H_i \right\}_{1\leq i \leq n}$ une famille de sous-groupe de $G$ tel que $H_iH_j = H_jH_i$ pour tout (i,j) tel que $1 \leq i < j \leq n$, $H_1H_2\ldots H_n $ est un sous-groupe de $G$. Vérifions que la propriété est vrai pour $n+1$.

    Soit une $\left\{ H_i \right\}_{1\leq i \leq n+1}$ une famille de sous-groupe de $G$ tel que $H_iH_j = H_jH_i$ pour tout (i,j) tel que $1 \leq i < j \leq n+1$, on a :

    \begin{align*}
        (H_1H_2\ldots H_{n-1}H_n)H_{n+1} &= H_1H_2\ldots H_{n-1}  H_{n+1} H_n \text{, par hypothèse,}  \\
        &= H_1H_2 \ldots H_{n+1}  H_{n-1}  H_n, \\
        &\vdots \\
        &= H_{n+1} (H_1H_2\ldots H_{n-1}H_n).
    \end{align*}

    De plus, par hypothèse de récurrence, $(H_1H_2\ldots H_{n-1}H_n)$ est un sous-groupe de G. 
    
    Ainsi, $H_1H_2\ldots H_nH_{n+1}$ est un sous-groupe de G.

    \item Conclusion :

    Le corrolaire est démontré.
\end{itemize}

\item Soit $I$ un ensemble non vide et $\left\{ H_i \right\}_{i\in I}$  une famille de sous-groupes d'un groupe abélien $G$. Montrons que le sous-groupe $\displaystyle G' \Sigma_{i\in I} H_i$ est en somme directe si et seulement si tout $x \in G'$ s'écrt de 
façon unique :
\[
x \in \displaystyle \sum_{i \in I} x_{i_k}.
\]

Supposons que chaque $z \in G'$ s'écrive de façon unique. Soit $Z \in \displaystyle \bigcap_{i \in I} H_i$, pour tout $j \in J$, on peut écrire :

\[z = \underbrace{z}_{\in H_j} + \underbrace{0}_{\in \sum_{i\neq j} H_i} = \underbrace{0}_{\in H_j} + \underbrace{z}_{\in \sum_{i\neq j} h_i}\]

$z$ s'écrivant de façon unique, on en conclut que $z = 0$, donc $G'$ est en somme directe.

\medskip

Supposons que $G'$ soit en somme directe, c'est-à-dire que $\displaystyle \bigcap_{i \in I} H_i = \left\{ 0 \right\}$. Soit $z \in G'$, supposons que $z$ s'écrive de deux façcons différentes :

\[ z = \displaystyle \sum_{i\in I} x_i =  \sum_{i\in I} x'_i \text{, avec au moins un i tel que } x_i \neq x_i'\]

pour tout $j \in I$, on peut écrire :

\[x_j - x_j' =  \displaystyle \sum_{\substack{i\in I
\\ i\neq j}} x'_i - \sum_{\substack{i\in I
\\ i\neq j}} x_i = \sum_{\substack{i\in I
\\ i\neq j}} x'_i - x_i.  \]

On a $x_j - x'j \in H_j$, et   $\displaystyle \sum_{\substack{i\in I
\\ i\neq j}} x'_i - x_i  \in \sum_{\substack{i\in I
\\ i\neq j}} H_i$, donc $\displaystyle x_j - x_j' \in H_j \cap \sum_{\substack{i\in I
\\ i\neq j}} x'_i - x_i.  = {0}$.

Donc, pour tout $j \in I$, z s'écrit de façon unique.

\end{abc}