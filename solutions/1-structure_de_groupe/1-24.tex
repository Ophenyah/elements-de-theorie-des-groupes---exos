\begin{abc}
    \item On sait la multiplication de matrice associative, et que $I$ est le neutre pour cette opération. Vérifions la fermeture et l'inversibilité :


\[
\begin{array}{c|cccc}
   \times & I & A & B & C \\ \hline
    I     & I & A & B & C \\ 
    A     & A & I & C & B \\
    B     & B & C & I & A  \\
    C     & C & B & A & I \\
\end{array}
\]

On a bien $K_1$ est un sous-groupe de $GL(2,\R)$. Faisons pareil pour $K_2$, dont on sait la loi associative :

\[
\begin{array}{c|cccc} 
    \times       & \overline{1} & \overline{3} & \overline{5} & \overline{7} \\ \hline
    \overline{1} & \overline{1} & \overline{3} & \overline{5} & \overline{7} \\ 
    \overline{3} & \overline{3} & \overline{1} & \overline{7} & \overline{5} \\
    \overline{5} & \overline{5} & \overline{7} & \overline{1} & \overline{3}  \\
    \overline{7} & \overline{7} & \overline{5} & \overline{3} & \overline{1} \\
\end{array}
\]

On vérifie bien que $K_2$ est un groupe.

\item Il n'existe à isomorphisme près que 2 groupes d'ordre 4, $\dfrac{\Z}{4\Z}$ et $\dfrac{\Z}{2\Z} \times \dfrac{\Z}{2\Z}$. Comme il n'existe pas d'élément d'ordre 4 dans $K_1$ ou $K_2$, on en déduit qu'ils sont isomorphes entre eux et au groupe de Klein.

\end{abc}