Soit $G$ d'ordre $2n$, définissons la relation d'équivalence :

\[x \mathcal{R} y \Leftrightarrow x=y \text{ ou } x = y^{-1}.\]

Soit $\{x_i\}_{i\in I}$ une famille de représentants des classes modulo $\mathcal{R}$.
On a $1 \leq \overline{x_1} \leq 2$. le groupe se partitionne en $k$ classes d'un élément et $l$ classes de deux éléments, et on a donc :

\[2n = k + 2l\]

Pour respecter la parité, il faut donc que k soit pair, et sachant que $k>1$, qu'il existe au moins un élément différent du neutre tel que $x^2=e$.