De façon immédiate, on a que $\gamma_1 \subset GL(2, \R )$, $\Gamma_2 \subset \C^*$ et $\Gamma_3 \subset \cycle{5}$. Ecrivons leur table de Cayley pour vérifier la stabilité et l'existence d'un unique inverse.


\[
\begin{array}{c|cccc}
   \times   & I        & \gamma_1 & \gamma_2 & \gamma_3 \\ \hline
   I        & I        & \gamma_1 & \gamma_2 & \gamma_3 \\    
   \gamma_1 & \gamma_1 & \gamma_2 & \gamma_3 & I \\
   \gamma_2 & \gamma_2 & \gamma_3 & I        & \gamma_1 \\
   \gamma_3 & \gamma_3 & I        & \gamma_1 & \gamma_2 \\ 
\end{array}
\quad
\begin{array}{c|cccc}
   \times & 1 & i & -1 & -i \\ \hline
    1     & 1 & i & -1 & -i \\
    i     & i & -1 & -i & 1 \\
    -1    & -1& -i & 1 & i \\
    -i    & -i & 1 & i & -1 \\ 
\end{array}
\quad
\begin{array}{c|cccc}
   \times & \overline{1} & \overline{2} & \overline{3} & \overline{4} \\ \hline
   \overline{1} & \overline{1} & \overline{2} & \overline{3} & \overline{4} \\
   \overline{2} & \overline{2} & \overline{4} & \overline{1} & \overline{3}  \\
   \overline{3} & \overline{3} & \overline{1} & \overline{4} & \overline{2} \\
    \overline{4} & \overline{4} & \overline 3 & \overline 2 & \overline 1 \\ 
\end{array}
\]

On remarque que ce sont tous des groupes cyclique d'ordre 4, avec $\Gamma_1 = \langle \gamma_1 \rangle$,
$\gamma_2 = \langle i \rangle$ et $\gamma_3 = \langle 2 \rangle$, ils sont donc tous isomorphes entre eux.