Chaque élément d'un groupe possède un unique inverse, l'application est donc trivialement bijective.

Supposons que $G$ soit abélien :

\begin{align*}
    \forall a,b \in G,\quad f(ab) &= (ab)^{-1}, \\
    &= b^{-1} a^{-1}, \\
    &= a^{-1} b^{-1}, \\
    &= f(a)f(b).
\end{align*}

Donc abélien $\Rightarrow$ automorphisme.

Supposons que $f$ soit un automorphisme :

\begin{align*}
    \forall a,b \in G,\quad f(ab)&=f(a)f(b), \\
    b^{-1}a^{-1} &= a^{-1}b^{-1}, \\
    ab &= ba.
\end{align*}

ainsi, $f$ est un automorphisme si et seulement si $G$ est abélien.